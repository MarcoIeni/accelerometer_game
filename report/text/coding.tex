\section{Introduction}
This document is the report of the Advanced Operating Systems project course of M.Sc. Computer Science and Engineering at Politecnico di Milano.

It is a development project that involves programming an STM32F4 Discovery kit
for STM32F407 in C/C++ by using \href{http://miosix.org/}{Miosix}, an OS kernel designed to run on 32bit microcontrollers.

Git was used as version control system and you can check the GitHub project repository
at \href{https://github.com/MarcoIeni/accelerometer_game}{this} link.
\subsection{Problem statement}
The goal of the project is to design a game for 2 to 4 players, where each player should tap the accelerometer once and then again after 5 seconds. The player who gets closer to 5 seconds wins.

Therefore, it is required to implement tap detection for the accelerometer and implement the game logic.
\subsection{Summary of the work}
The project was completed accomplishing all the tasks of the problem and we
always worked together doing pair-programming.

We used no external libraries except for Miosix obviously.

We asked for the help of our advisor two times, because we had a few
problems with the initialization of the SPI.
% \begin{enumerate}
%   \item the first time we made the mistake of using the software SPI provided by
%     miosix while, in order to communicate with the accelerometer, the hardware
%     SPI was necessary
%   \item the second time we had a correct initialization of the hardware SPI,
%     except for the frequency, that was too high with respect to the one
%     supported by the accelerometer of the board.
% \end{enumerate}

We present here our git flow, because we think that it represents a good summary
of our work.

The main functionalities of our projects were divided into different branches:
\begin{description}
  \item [players\_select:] here we did the part of the program that is
    responsible to choose the number of players that will play the game.
  \item [play\_logic:] here we developed the game logic of the five seconds
    challenge by using the user button instead of the accelerometer at first.
  \item [accelerometer:] in this branch we tested the communication with the
    accelerometer of the board, by reading the who\_am\_i register and the by
    programming the state machine that decides when to send the interrupt for
    the click detection.
\end{description}

These three branches were merged with the master one in this order. Then, we
created another branch called \textbf{interrupts\_logic} to develop the handling of the two different interrupts
of our system: the one that comes from the user button that manages the player
selection, and the one that comes from the accelerometer that manages the tap
detection for the 5 seconds challenge.

Finally, this branch was merged with the master which, at this point, represents our final release.

The work started on 27 Feb and ended on May.

\pagebreak

\section{The Game}
The design of the game is very simple, which is also due to the nature of the game itself.

We associated the position of the four leds to the position of the players, that
ideally should be seated all around the microcontroller.
In this way when a led blinks or is on, you can understand to which player we
are referring to.

\subsection{Number of players selection}
When the microcontroller is powered on, the game starts and two leds are on.
The number of leds indicate the number of the players and the position in which
they should be seated.
At this point press the user button (the blue one) in order to select the right
number of players.

Finally, in order to start the 5 seconds challenge, you should confirm the number of players by tapping on the board.

\subsection{Five seconds challenge}
The first led is on, and the player which is seated in that position has to
tap on the board the first time, wait for a time which is closer as much as
possible to 5 seconds and then tap on the board again.

The board at this point stores the difference in time between the two taps and
switches on another led, which indicates the next player that has to play.

When all the players participated, then the led that corresponds to the player
which was closed to 5 seconds starts toggling and the five seconds challenge start
the game.

If you want to go back to the selection of the number of players you need to
press the reset button (the black one).

\hypertarget{installation}{%
\subsection{Installation}\label{installation}}

Miosix kernel is used as submodule as specified in
\href{https://miosix.org/wiki/index.php?title=Miosix_and_git_workflow\#Setting_up_an_out_of_git_tree_project}{this}
wiki page.
In order to build the project, first clone the repository.

\hypertarget{submodule-initialization}{%
\subsubsection{Submodule
Initialization}\label{submodule-initialization}}

Initialize the submodule:

\begin{verbatim}
git submodule init
\end{verbatim}

If your git authentication method is https instead of ssh you need to
open \texttt{.git/config} with your favourite text editor and inside
\texttt{submodule\ "miosix-kernel"} change the url to
\texttt{https://github.com/fedetft/miosix-kernel.git}

\hypertarget{submodule-update}{%
\subsubsection{Submodule Update}\label{submodule-update}}

Update the submodule:

\begin{verbatim}
git submodule update --remote --rebase
\end{verbatim}

\section{Design and implementation}
\subsection{Main modules}
We assigned each main task of our system to a different component:

\begin{description}
  \item [lis3dsh\_manager:] it handles the initialization of the spi and the
    accelerometer. It also defines the functions to communicate with the accelerometer through
    the spi.
  \item [play\_logic:] it allows to select the number of players and handles the logic of the game.
  \item [led\_manager:] it handles the initialization of the leds and provides an
    abstraction layer that allows to switch on and off the leds starting from
    the player number.
  \item [interrupt\_manager:] it is responsible for the management of the
    interrupts, in particular the one that come from the press of the button and
    the one that comes from the click of the accelerometer.
\end{description}

\subsection{Accelerometer configuration}
In addition to the datasheet of the LIS3DSH, we used the application note
document \cite{note} which was very useful to determine the state machine to be used.

In this document, in fact, it was present a state machine already configured for
double-tap detection.
Therefore we adjusted it in order to implement the single-tap detection.

After an analysis of the already made state machine, we cut off the states that
went from the recognition of the second tap till the end.

\section{Conclusions and Future Works}
We adjusted the parameters of the accelerometer configuration based only
on our experience with the bare board.
In the future, it would be interesting to test this configuration on a board
inside a package and to calibrate the parameters also for this scenario.